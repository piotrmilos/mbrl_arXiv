
% if you need to pass options to natbib, use, e.g.:
% \PassOptionsToPackage{numbers, compress}{natbib}
% before loading nips_2018

% ready for submission
% \usepackage{nips_2018}

% to compile a preprint version, e.g., for submission to arXiv, add
% add the [preprint] option:
% \usepackage[preprint]{nips_2018}

% to compile a camera-ready version, add the [final] option, e.g.:
% \usepackage[final]{nips_2018}

% to avoid loading the natbib package, add option nonatbib:
% \usepackage[nonatbib]{nips_2018}

\usepackage[utf8]{inputenc} % allow utf-8 input
\usepackage[T1]{fontenc}    % use 8-bit T1 fonts
% \usepackage[colorlinks=true, linkcolor=black, citecolor=black, filecolor=black, urlcolor=black]{hyperref}       % hyperlinks
\usepackage{url}            % simple URL typesetting
\usepackage{booktabs}       % professional-quality tables
\usepackage{amsfonts}       % blackboard math symbols
\usepackage{amsmath}
\usepackage{nicefrac}       % compact symbols for 1/2, etc.
\usepackage{microtype}      % microtypography

% \usepackage{algorithm2e}
% \usepackage[noend]{algpseudocode}
\usepackage[ruled, vlined, linesnumbered]{algorithm2e}
\algsetup{linenosize=\small}
% \usepackage{etoolbox}\AtBeginEnvironment{algorithm}{\small}

\usepackage{todonotes}

\newcommand{\env}{{\texttt{env}}}
\newcommand{\Env}{{\texttt{Env}}}
\newcommand{\Aa}{{\mathcal A}}
\newcommand{\Rr}{{\mathcal R}}
\newcommand{\Oo}{{\mathcal O}}
\newcommand{\Dd}{{\mathcal D}}
\newcommand{\Prob}{{\mathcal P}}

\newcommand{\pong}{\texttt{Pong}}
\newcommand{\breakout}{\texttt{Breakout}}
\newcommand{\freeway}{\texttt{Freeway}}
\newcommand{\bankh}{\texttt{Bank Heist}}
\newcommand{\boxing}{\texttt{Boxing}}
\newcommand{\bowling}{\texttt{Bowling}}
\newcommand{\asterix}{\texttt{Asterix}}
\newcommand{\seaquest}{\texttt{Seaquest}}
\newcommand{\hero}{\texttt{Hero}}
\newcommand{\crazyclimber}{\texttt{Crazy Climber}}
\newcommand{\kungfumaster}{\texttt{Kung Fu Master}}
\newcommand{\atlantis}{\texttt{Atlantis}}
\newcommand{\battlezone}{\texttt{Battle Zone}}
\newcommand{\privateeye}{\texttt{Private Eye}}


\usepackage[font={footnotesize,it}]{caption}

\newtheorem{definition}{Definition}

\usetikzlibrary{shapes.geometric,backgrounds,
  positioning-plus,node-families,calc}
\tikzset{
  basic box/.style = {
    shape = rectangle,
    align = center,
    draw  = #1,
    fill  = #1!25,
    rounded corners},
  header node/.style = {
    Minimum Width = header nodes,
    font          = \strut\Large\ttfamily,
    text depth    = +0pt,
    fill          = white,
    draw},
  header/.style = {%
    inner ysep = +1.5em,
    append after command = {
      \pgfextra{\let\TikZlastnode\tikzlastnode}
      node [header node] (header-\TikZlastnode) at (\TikZlastnode.north) {#1}
      node [span = (\TikZlastnode)(header-\TikZlastnode)]
        at (fit bounding box) (h-\TikZlastnode) {}
    }
  },
  hv/.style = {to path = {-|(\tikztotarget)\tikztonodes}},
  vh/.style = {to path = {|-(\tikztotarget)\tikztonodes}},
  fat blue line/.style = {ultra thick, blue}
}

\usetikzlibrary{arrows,decorations.pathmorphing,backgrounds,positioning}

\definecolor{echoreg}{HTML}{2cb1e1}
\definecolor{olivegreen}{rgb}{0,0.6,0}
\definecolor{mymauve}{rgb}{0.58,0,0.82}

\usepackage{etoolbox}

\newtoggle{redraw}
\newtoggle{redraw2}

\usepackage{wrapfig}

\tikzset{%
pics/cube/.style args={#1/#2/#3/#4}{code={%
	\begin{scope}[line width=#4mm]
	\begin{scope}
	\clip (-#1,-#2,0) -- (#1,-#2,0) -- (#1,#2,0) -- (-#1,#2,0) -- cycle;
	\filldraw (-#1,-#2,0) -- (#1,-#2,0) -- (#1,#2,0) -- (-#1,#2,0) -- cycle;
	\end{scope}
\iftoggle{redraw}{%
}{%
	\begin{scope}
	\clip (-#1,-#2,0) -- (-#1-#3,-#2,-#3) -- (-#1-#3,#2,-#3) -- (-#1,#2,0) -- cycle;
	\filldraw (-#1,-#2,0) -- (-#1-#3,-#2,-#3) -- (-#1-#3,#2,-#3) -- (-#1,#2,0) -- cycle;
	\end{scope}
}
\iftoggle{redraw2}{%
}{
	\begin{scope}
	\clip (-#1,#2,0) -- (-#1-#3,#2,-#3) -- (#1-#3,#2,-#3) -- (#1,#2,0) -- cycle;
	\filldraw (-#1,#2,0) -- (-#1-#3,#2,-#3) -- (#1-#3,#2,-#3) -- (#1,#2,0) -- cycle;
	\end{scope}
}
	\node[inner sep=0] (-A) at (-#1-#3*0.5, 0, -#3*0.5) {};
	\node[inner sep=0] (-B) at (#1-#3*0.5, 0, -#3*0.5) {};
	
	\coordinate (-V) at (#1, #2);
	\coordinate (-W) at (#1, -#2);
	\end{scope}
}}}

% macros for recurrent simulators

\usepackage{amsmath}

\renewcommand{\figref}[1]{Fig. \ref{#1}}
\renewcommand{\secref}[1]{Sec. \ref{#1}}
\newcommand{\appref}[1]{Appendix \ref{#1}}
\definecolor{myred}{cmyk}{0,0.9,0.9,0.1}
\definecolor{myblue}{cmyk}{0,0.3,0.9,0.1}
\newcommand{\myred}{\color{myred}}
\newcommand{\myblue}{\color{myblue}}
\newcommand{\PDT}{PDT}
\newcommand{\ODT}{ODT}
\newcommand{\OD}{observation-dependent} 
\newcommand{\PD}{prediction-dependent} 
\setlength{\bibsep}{0.2pt}

\newcommand{\myvec}[1]{\mathbf{#1}}
\renewcommand{\va}{\myvec{a}}
\renewcommand{\vc}{\myvec{c}}
\renewcommand{\vf}{\myvec{f}}
\renewcommand{\vh}{\myvec{h}}
\renewcommand{\vi}{\myvec{i}}
\renewcommand{\vo}{\myvec{o}}
\renewcommand{\vs}{\myvec{s}}
\renewcommand{\vv}{\myvec{v}}
\renewcommand{\vx}{\myvec{x}}
\renewcommand{\vy}{\myvec{y}}
\renewcommand{\vz}{\myvec{z}}
\newcommand{\vW}{\myvec{W}}
\newcommand{\ha}{\vv}

\newcommand{\fix}{\marginpar{FIX}}
\newcommand{\new}{\marginpar{NEW}}

\newcommand{\ie}{\emph{i.e.}}
\newcommand{\eg}{\emph{e.g.}}


%\usepackage{caption}
% \captionsetup{font=footnotesize}

\makeatletter
\newcommand{\removelatexerror}{\let\@latex@error\@gobble}
\makeatother