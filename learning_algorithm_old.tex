\section{Model based learning algorithm}
\label{subsec:model}

Reinforcement learning is formalized by Markov decision processes (MDP). An MDP is defined as a tuple $(\mathcal{S}, \Aa, P, r, \gamma)$, where $\mathcal{S}$ is a state space, $\Aa$ is a set of actions available to an agent, $P$ is the transition kernel, $r$ is reward function and $\gamma\in (0,1)$ is the discount factor. 

In this work we refer to MDPs as environments and assume that environments fulfill the following assumptions:
(1) Our environments do not have direct access to the state (i.e., the RAM of Atari 2600 emulator). Instead we use visual observations, typically $210\times 160$ RGB images. A single image does not determine the state. To circumvent this we stack the four previous frames. After this augmentations we will assume that the observation can be identified with the state.
%%SL.1.15: can probably remove that last sentence -- since the model is recurrent anyway, I'm not sure we even need this assumption
(2) Our environments are deterministic, namely the transition kernel $P$ can be identified with a function $env:\mathcal{S}\times \Aa\mapsto \mathcal{S}$, i.e. given a state and action the environment produces the next state.
%%SL.1.15: do we actually need this assumption?
(3) The action space $\Aa$ and the space of rewards are finite. 

A reinforcement learning agent interacts with the MDP issuing actions according to a policy. Formally, policy $\pi$ is a mapping from states to probability distributions over $\mathcal{A}$. The quality of a policy is measured by the value function $\mathbb{E}_{\pi}\left(\sum_{t=0}^{+\infty}\gamma^t r_{t+1}|s_0=s \right)$, which for a starting state $s$ estimates the total discounted reward gathered by the agent. In Atari 2600 games our goal is to find a policy which maximizes the value function when the starting state is the beginning of the game.

Crucially, apart from a Atari 2600 emulator environment $env$ we will use \textit{a neural network simulated environment} $env'$. The environment $env'$ shares the action and reward spaces with $env$ and produces visual observations in the same format. The  environment $env'$ will be trained to mimic $env$.  The principal aim of our work is to check
%%SL.1.15: well, maybe not just to check -- perhaps we can say The principal aim of our work is to train a policy...
if one can 
train a policy $\pi$ using a simulated environment $env'$ so that $\pi$ achieves a desired performance in the original environment $env$. During the training we aim to use as little interactions with $env$ as possible. 


The initial data to train $env'$ comes from random rollouts of $env$. As this is unlikely to capture all aspects of the environment, thus we use the following data-aggregation iterative algorithm inspired by Section 4 in  \cite{trpo_ensemble} and Section 5 in \cite{world_models}.
%%SL.1.15: To be fair, this basic scheme has been in use for decades and is basically ubiquitous for model-based RL. Feels odd to credit it to two papers from the last couple of years.

%\begin{figure*}[H]
\begin{algorithm}[H]
\caption{Pseudocode for model-based RL}\label{dpll}
\begin{algorithmic}
\STATE Initialize policy $\boldsymbol\pi$
\STATE Initialize model parameters $env'$
\STATE Initialize empty set $\mathbf{D}$
\WHILE{not done} 
\STATE \underline{$\triangleright$ collect observations from real env.}
\WHILE{not enough observations}
\STATE $(\mathbf{state},\mathbf{reward}) \gets env(\boldsymbol\pi(\mathbf{state}))$
\STATE $\mathbf{D} \gets \mathbf{D} \cup (\mathbf{state},\mathbf{reward})$
\ENDWHILE
\STATE \underline{$\triangleright$ update model using collected data.}
\STATE $\boldsymbol\theta \gets \text{TRAIN\_SUPERVISED}(env`, \mathbf{D})$
\STATE \underline{$\triangleright$ update policy using world model.}
\STATE $\boldsymbol\pi \gets \text{TRAIN\_RL}(\boldsymbol\pi, \boldsymbol\theta)$
\ENDWHILE
\end{algorithmic}
\label{basic_loop}
\label{alg:basic_loop}
\end{algorithm}
% \end{figure*}